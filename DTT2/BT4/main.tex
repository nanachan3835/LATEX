\documentclass{article}
\usepackage{graphicx} % Required for inserting images
\usepackage[utf8]{vietnam}
\usepackage{amsmath}
\usepackage{setspace}
\renewcommand{\baselinestretch}{1.2}
\usepackage{graphicx}
\usepackage{tocbibind} 
\usepackage{hyperref}


\hypersetup{
    colorlinks=true, % Kích hoạt màu cho liên kết
    linkcolor=black, % Màu liên kết nội bộ (trong tài liệu)
    urlcolor=black,  % Màu liên kết trang web
    citecolor=black, % Màu liên kết trích dẫn
    pdfborder={0 0 0}, % Không có viền
}


















\begin{document}



% Trang bìa
\begin{titlepage}
    \centering
    \vspace{0.5cm}

    \LARGE
    Trường Đại Học Bách Khoa Hà Nội

    \LARGE
    \textbf{VIỆN ĐIỆN TỬ - VIỄN THÔNG}
    
%\vspace{1.2cm}
%\includegraphics[width=0.5\textwidth]{Screenshot from 2024-10-29 21-01-20.png} % Thay %logo.png bằng tên tệp hình ảnh của bạn

\begin{figure}[ht]
    \centering
    \includegraphics[width=0.8\textwidth]{LOGO.png}
    \label{fig:ten_label}
\end{figure}


    \vspace{1cm}

    \textbf{BÁO CÁO}

    \vspace{0.2cm}
    \LARGE
    \textbf{\fontsize{19}{24}\selectfont DESIGN LOW-PASS FILTER}

    \vspace{0.2cm}
    \LARGE

    \begin{center}
        
    
    \textbf{Họ và Tên: Dương Quốc Dũng }
    
    \textbf{MSSV: 20213835}
    
    \end{center}
    
   

\end{titlepage}



\tableofcontents
\listoffigures
\newpage

\section{CƠ SỞ LÝ THUYẾT}
\subsection{Bộ lọc là gì}

Bộ lọc là các mạch điện tử hoặc mạng xử lý tín hiệu, làm thay đổi tín hiệu theo
cách phụ thuộc vào tần số của chúng. Cơ chế của bộ lọc dựa trên tính chất phụ thuộc tần
số của trở kháng trong tụ điện và cuộn cảm.

Có ba loại bộ lọc chính là:

\begin

\subsection{Các tham số cần chú ý}

\subsection{Các đáp ứng tiêu chuẩn của bộ lọc}














\end{document}