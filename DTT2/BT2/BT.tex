\documentclass{article}
\usepackage{graphicx} % Required for inserting images
\usepackage[utf8]{vietnam}
\usepackage{amsmath}
\usepackage{setspace}
\renewcommand{\baselinestretch}{1.2}
\usepackage{graphicx}
\usepackage{tocbibind} 
\usepackage{hyperref}


\hypersetup{
    colorlinks=true, % Kích hoạt màu cho liên kết
    linkcolor=black, % Màu liên kết nội bộ (trong tài liệu)
    urlcolor=black,  % Màu liên kết trang web
    citecolor=black, % Màu liên kết trích dẫn
    pdfborder={0 0 0}, % Không có viền
}


















\begin{document}



% Trang bìa
\begin{titlepage}
    \centering
    \vspace{0.5cm}

    \LARGE
    Trường Đại Học Bách Khoa Hà Nội

    \LARGE
    \textbf{VIỆN ĐIỆN TỬ - VIỄN THÔNG}
    
%\vspace{1.2cm}
%\includegraphics[width=0.5\textwidth]{Screenshot from 2024-10-29 21-01-20.png} % Thay %logo.png bằng tên tệp hình ảnh của bạn


    \vspace{1cm}

    \textbf{BÁO CÁO}

    \vspace{0.2cm}
    \LARGE
    \textbf{\fontsize{19}{24}\selectfont Tính toán phối hợp trở kháng chữ L}

    \vspace{0.2cm}
    \LARGE

    \begin{center}
        
    
    \textbf{Họ và Tên: Dương Quốc Dũng}
    
    \textbf{MSSV: 20213835}
    
    \end{center}
    
   

\end{titlepage}



\tableofcontents
\listoffigures
\newpage






\section{Cơ sở lý thuyết chuyển đổi từ song song sang nối tiếp cho L và R}

\begin{equation}
\Large
       Z_{in} = Z_{c} // (Z_{l} nt R_{l}) => R_{l}+Z_{l} = R_{c}
\end{equation}

Ta có 

\begin{equation}
\Large
       R_{s} + s.L_{s} = \frac{R_{P}.L_{P}.s }{R_{P}+L_{P}}
\end{equation}

Khai triển biểu thức trên 
\begin{equation}
\Large
       (R_{s} + s.L_{s}) x (R_{P}+L_{P}) = R_{P}.L_{P}.s 
\end{equation}


\begin{equation}
\Large
       (R_{s} + jw.L_{s}) x (R_{P}+L_{P}) = R_{P}.L_{P}.jw 
\end{equation}

Từ đó ta có hệ phương trình 
\begin{equation}
\Large
    R_{s}.R_{p} - L_{s}.L_{p}.w^2 =0 
\end{equation}

\begin{equation}
\Large
    R_{s}.L_{p}.jw - L_{s}.R_{p}.jw =   R_{P}.L_{P}.jw
\end{equation}

Ta có (6) rút gọn :


\begin{equation}
\Large
    Q_{s} = Q_{p} => \frac{Lw}{R_{s}} = \frac{R_{p}}{L_{p}w}
\end{equation}

\begin{equation}
\Large
    R_{s}.L_{p}.w - L_{s}.R_{p}.w =   R_{P}.L_{P}.w
\end{equation}


Biến đổi biểu thức (8) theo (7) ta được:

\begin{equation}
\Large
1+ \frac{L_{s}.w}{R_{s}} . \frac{R_{p}}{L_{p}.w}  = \frac{R_{p}. L_{p}.w}{R_{s}.L_{p}.w} 
\end{equation}

\begin{equation}
\Large
1+ Q^2 = \frac{R_{p}}{R_{s}} 
\end{equation}

Biến đổi tương tự ta có :


\begin{equation}
\Large
1+ \frac{1}{Q^2} = \frac{L_{p}}{L_{s}} => L_{p} = L_{s} . \frac{Q^2}{Q^2+1}
\end{equation}






\section{Bài tập với các trường hợp mạch Phối hợp trở khảng chữ L với các trường hợp a,b,c}

\subsection{Trường hợp b}
Cho đề bài RL = 50 ohm , Zin = 25 ohm , F = 5 Ghz, Re{Zin} < Rl 
Có Rl > Re{Zin} , ta dùng phương pháp PHTK , như sau :



Áp dụng biến đổi trở khảng  thụ động cho L1 và Rl ở mạch // ta có mạch NT:


% \begin{figure}[ht]
%     \centering
%     \includegraphics[width=0.8\textwidth]{Screenshot from 2024-10-29 20-07-22.png}
%     \caption{LP->LS}
%     \label{fig:ten_label}
% \end{figure}





Hệ phương trình như sau 

\begin{equation}
\Large
Q_{p} = Q_{s} = Q
\end{equation}

\begin{equation}
\Large
L_{p} = L_{s} . \frac{Q^2}{Q^2+1}(L1=Lp)
\end{equation}

\begin{equation}
\Large
 R_{p} = R_{s} (Q^2+1)
\end{equation}

Từ mạch b, Ta có:

\begin{equation}
\Large
Z_{in} = R_{s} + Z_{Ls} + Z_{C1} = \frac{R_{p}}{Q^2+1} + L_{p}.\frac{Q^2+1}{Q^2} + Z_{C1} 
\end{equation}

\begin{equation}
\Large
\frac{R_{p}}{Q^2+1} = Re{Z_{in}} =>  Q = 1 = \frac{R_{p}}{Lw} => R_{l} = Lw
\end{equation}

\begin{equation}
\Large
=> L1 = \frac{R_{l}}{w} = \frac{50}{2.\pi.5.10^6} = 1,59 . 10^-6
\end{equation}


\begin{equation}
\Large
\frac{1}{jC_{1}w_{1}} = -jwL_{s} => C_{1} = \frac{1}{w^2.L_{s}} => C_{1} = 31.86 (pF)
\end{equation}




\subsection{Trường hợp c}
Ta có Zin = 100 ohm , F = 56 hz , Rl = 50 ohm 
Có Rl < Re{Zin} , ta dùng phương pháp PHTK , như sau :
Áp dụng biến đổi trở khảng  thụ động cho L1 và C1 ở mạch (1), ta có (2):

% \begin{figure}[ht]
%     \centering
%     \includegraphics[width=0.8\textwidth]{Screenshot from 2024-10-29 20-08-12.png}
%     \caption{LS->LP}
%     \label{fig:ten_label}
% \end{figure}


Hệ phương trình như sau 

\begin{equation}
\Large
Q_{p} = Q_{s} = Q
\end{equation}

\begin{equation}
\Large
L_{p} = L_{s} . \frac{Q^2}{Q^2+1}(L_{s} = L_{1})
\end{equation}

\begin{equation}
\Large
 R_{p} = R_{s} (Q^2+1) (R_{s} = R_{1})
\end{equation}

Từ mạch b ta có: 

\begin{equation}
\Large
 \frac{1}{Z_{in}} = \frac{1}{R_{p}} + \frac{1}{L_{p}.jw} + jw.C_{1}
\end{equation}


\begin{equation}
\Large
Re{\frac{1}{Z_{in}}} = \frac{1}{R_{p}} => \frac{1}{100} = \frac{1}{50.(Q^2+1)} => Q_{s} = 1
\end{equation}

\begin{equation}
\Large
=> \frac{Lw}{R_{l}} = 1 => L_{1} = \frac{50}{2.\pi.5.10^6} = 1,59 . 10^-6 
\end{equation}

\begin{equation}
\Large
=> L_{p} = 2.L_{s} = 2.L_{1} = 3,18.10^-6 (H)
\end{equation}


\begin{equation}
\Large
Lại có :
\frac{1}{L_{p}.jw} + jw.C_{1} = 0 => w^2.C_{1}.L_{p} = 1 => C_{1} = 31.86 pF
\end{equation}


\subsection{Trường hợp D}
Zin = 100 ohm , Rl= 50 ohm, F= 5GHz
Có Rl < Re{Zin} , ta dùng phương pháp PHTK , như sau :

Áp dụng biến đổi trở khảng  thụ động cho L1 và C1 ở mạch (1), ta có (2):
% \begin{figure}[ht]
%     \centering
%     \includegraphics[width=0.8\textwidth]{add.png}
%     \caption{Cs->Cp}
%     \label{fig:ten_label}
% \end{figure}
Hệ phương trình như sau 

\begin{equation}
\Large
Q_{p} = Q_{s} = Q
\end{equation}

\begin{equation}
\Large
C_{p} = C_{s} . \frac{Q^2}{Q^2+1} (C_{s} = C_{1})
\end{equation}

\begin{equation}
\Large
 R_{p} = R_{s} (Q^2+1) (R_{s} = R_{1})
\end{equation}


Từ mạch b ta có: 

\begin{equation}
\Large
 \frac{1}{Z_{in}} = \frac{1}{R_{p}} + \frac{1}{L_{1}.jw} + jw.C_{p}
\end{equation}


\begin{equation}
\Large
Re{\frac{1}{Z_{in}}} = \frac{1}{R_{p}} => \frac{1}{100} = \frac{1}{50.(Q^2+1)} => Q_{s} = 1
\end{equation}

\begin{equation}
\Large
=> \frac{1}{C_{1}w} = R_{s} => C_{1} = \frac{1}{w.R_{l}} = 6,36. 10^-10 =63.6(pF)
\end{equation}


\begin{equation}
\Large
=> C_{p} = \frac{1}{2} . C_{s} => C_{p} = 31,8 (pF)
\end{equation}

Lại có:
\begin{equation}
\Large
=> \frac{1}{L_{1}.jw} + jw.\phi =0 => 1 = w^2.L_{1}.C_{p} => L_{1} = \frac{1}{w^2.C_{p}} = 3.18 (\mu H)
\end{equation}

\end{document}







