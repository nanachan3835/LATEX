\documentclass{article}
\usepackage{graphicx} % Required for inserting images
\usepackage[utf8]{vietnam}
\usepackage{amsmath
}
\usepackage{setspace}
\renewcommand{\baselinestretch}{1.2}


\title{DTTT2-BT1}
\author{DƯƠNG QUỐC DŨNG - 20213835}
\date{October 2024}













\begin{document}

\maketitle

\section{Đề bài}

Một bộ khuếch đại có công suất điểm nén 1dB đầu vào là -10 dBm và hệ số khuếch đại công suất Gain là 15 dB. Mắc nối tiếp bộ khuếch đại nói trên với 1 mixer có công suất điểm nén 1dB là 0 dBm. Tính công suất điểm nén 1dB của hệ thống (theo dBm) và giải thích kết quả. 

\section{Bài giải}
Biến đổi công thức,
Ta có công thức phi tuyến với điểm chặn bậc 3:

\begin{equation}
\Large
       \frac{1}{A^2}  = \frac{1}{A_{ip31}^2}  + \frac{\alpha_1^2}{A_{ip32}^2}   
\end{equation}

Ta lại có công thức điểm nén 1dB:
\begin{equation}
\Large
       Ạ_{1dB} = \sqrt{0.145 \times  |\frac{\alpha_1}{\alpha_2}|  }   
\end{equation}

Và công thức điểm chặn bậc 3:
\begin{equation}
    \Large
    A_{ip3} = \sqrt{ \frac{4}{3} \times  |\frac{\alpha_1}{\alpha_2}| }
\end{equation}

Vì vậy, từ (3) và (4) ta có thể  suy ra rằng:
\begin{equation}
    \Large
    A_{ip3} \propto A_{1dB}
\end{equation}

Ta lại có công thức của công suất điểm nén 1dB:
\begin{equation}
    \Large
    P_{1dB} = \frac{A_{1dB}^2}{4R}
\end{equation}

Từ (5) và (6), ta có thể suy ra:
\begin{equation}
    \Large
    P_{1dB} \propto A_{ip3}^2
\end{equation}

Từ (7) ta biến đổi (2), ta có phương trình mới như sau:
\begin{equation}
    \Large
    \frac{1}{P_{1dB System}}  = \frac{1}{P_{1dB Out}}  + \frac{A_1}{P_{1dB Mixer}}   
\end{equation}

Chuyển đổi công suất :
\begin{align}
    \LARGE 
    P_{1db,out} = -10dBm = 10^\frac{-10}{10} mW = 0,1 mW \\
    P_{1dB,mixer} = 0dBm = 10^\frac{0}{10} mW = 1 mW \\
    A_1 = 15db = 10^\frac{15}{10} lần 
\end{align}

Thay số vào công thức:
\begin{equation}
    \Large
    \frac{1}{P_{1dBSystem}}  = \frac{1}{0,1}  + \frac{10^\frac{15}{10}}{1}
\end{equation}
Ta được kết qủa :
\begin{equation}
    \Large
    P_{1dBSystem} = 0,024mW = 10 \times \log{0.024} = -16 dBm
\end{equation}



\end{document}
