\documentclass{article}
\usepackage{graphicx} % Required for inserting images
\usepackage[utf8]{vietnam}
\usepackage{amsmath}
\usepackage{setspace}
\renewcommand{\baselinestretch}{1.2}
\usepackage{graphicx}
\usepackage{tocbibind} 
\usepackage{hyperref}




\begin{document}

Ta có : I thuộc O1O2 

Vị trí của $N = \left(k + \frac{1}{2}\right) \lambda$, $I = k \lambda$, mà $N$ và $I$ là cực tiểu và cực đại gần nhau $\implies k$ là như nhau $\implies NI = \left(k + \frac{1}{2}\right) \lambda - k \lambda = \frac{\lambda}{2} = 0.625$.

$\implies \lambda = 1.25$.

Mà $M$ là cực đại nhất và gần $O_1 O_2$ nhất $O_2$.

Trước tiên, ta tìm các cực tiểu trên đường thẳng $O_1 O_2$. Vì sóng giao thoa xét trong độ dài khoảng $O_1 O_2$, nên ta có thể suy ra độ dài $O_1 O_2$ luôn lớn hơn độ dài từ $O_1$ đến các cực tiểu trong khoảng $O_1 O_2$.

Gọi độ dài $O_1 O_2 = L = 16$. Ta có:

\[
L > \left| \left(K + \frac{1}{2}\right) \lambda \right| \implies -\frac{L}{\lambda} + \frac{1}{2} < K < \frac{L}{\lambda} - \frac{1}{2}
\]

Số điểm cực tiểu sẽ là:

\[
-12.3 < K < 12.3 \implies \text{số điểm cực tiểu là } 24 \text{ điểm}.
\]

Vậy điểm cực tiểu thuộc đường tròn tâm $O_1$ và gần với $O_1 O_2$ nhất sẽ là điểm có $K = -12$.

Khi đó:

\[
d_1 - d_2 = \left(K + \frac{1}{2}\right) \lambda = \left|(-12 + \frac{1}{2}) \cdot 1.25\right| = 14.375.
\]

Lại có $d_1 = O_1 G = R = 16$.

\[
d_2 = d_1 - 14.375 = 16 - 14.375 = 1.625.
\]

Vậy khoảng cách từ $M$ đến $O_2$ ngắn nhất là $1.625$.


\end{document}